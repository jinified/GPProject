\def\year{2016}
\documentclass[letterpaper]{article}

\usepackage{aaai}
\usepackage{relsize}
\usepackage{sectsty}
\usepackage{amsmath}
\usepackage{amssymb}
\usepackage[utf8]{inputenc}
\usepackage[english]{babel}
\usepackage{titlesec}
\usepackage{graphicx} 
\usepackage{wrapfig}
\usepackage{indentfirst}
\usepackage[style=numeric,backend=bibtex]{biblatex}
% For aaai
\usepackage{times}
\usepackage{helvet}
\usepackage{courier}
\setlength{\pdfpagewidth}{8.5in} 
\setlength{\pdfpageheight}{11in}

\setcounter{secnumdepth}{1} % for section number
\addbibresource{reference.bib}

\begin{document}

	\title{Modelling Crimes using Gaussian Processes}
	\author{CS4246 Project Part 2: Active Learning and Bayesian Optimization\\ \\
	\bf \small Group 5:\\
	\small Choo Boon Yong Martin, A0132760M\\
	\small Leow Yijin, A0131891E\\
	\small Tan Soon Jin, A0112213E\\
	\small Teo Qi Xuan, A0124206W\\
	\small Won Jun Ru Daphne, A0126172M\\
	\small Zhu Liang, A0093910H\\
	}	
	
	\pdfinfo{
		/Title Active Learning and Bayesian Optimization
		/Author Martin, Yijin, Soon Jin, Qi Xuan, Daphne, Zhu Liang
	}
	
	\maketitle
	\thispagestyle{empty}
	\pagestyle{empty}
	
	
	%%%%%%%%%%%%%%%%%%%%%%%%%%%%%%%%%%%%%%%%%%%%%%%%%%%%%%%%%%%%%%%%%%%%%%%%%%%%%%%%
	
	\begin{abstract}
	\begin{quote}
	Crime is 

	\end{quote}
	\end{abstract}
	
	%%%%%%%%%%%%%%%%%%%%%%%%%%%%%%%%%%%%%%%%%%%%%%%%%%%%%%%%%%%%%%%%%%%%%%%%%%%%%%%%
	
	%%%%%%%%%% Section 1
	\section{Introduction}
	
	In part 1, our paper make use of Gaussian Processes (GP) to model multiple forms of heat maps of crime density in the District of Columbia (DC).
	In this paper, we will further look into the use of active learning to cut computation cost and to produce similar results with lesser data.\\ \\

	This paper will discuss the technical aspect, looking into the optimization criterion and qualitative advantages. 
	

	%%%%%%%%%% References

	\printbibliography

\end{document}